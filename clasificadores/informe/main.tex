\documentclass[12pt, a4paper]{report}
\usepackage[utf8]{inputenc}
\usepackage[spanish]{babel}
\usepackage{float}
\usepackage{geometry} % márgenes
\usepackage[conEntregas]{caratula} % carátula

\begin{document}

\newgeometry{margin=3cm} % para que entre todo en una hoja

\subtitulo{Trabajo Práctico 1}
\titulo{Regresión Polinomial}

\fecha{\today}

\materia{Reconocimiento De Patrones}

\integrante{Ingaramo, Pablo}{544/15} {pablo2martin@hotmail.com}
\integrante{Cuba, Kennedy}{503/15}{kennedy.wrc@hotmail.com}

% Pongan cuantos integrantes quieran

\maketitle

\restoregeometry

\tableofcontents{}
\begin{abstract}
    El objetivo de este trabajo práctico es ver como se puede utilizar
la regresión polinomial para estimar un conjunto de datos mediante
polinomios. \par
\indent Para tener una forma de medir y comparar los resultados, 
se presentán funciones de error y se analiza como 
afecta al vector que los minimiza. \par 
\indent Luego se presentará una conocida
set datos ("Boston Housing Dataset") y la forma de calcular la matriz de
correlación, junto con un breve analizasis de la feature MEDV
del set datos.
\end{abstract}

\chapter{Introducción teórica} 


\section{Regresión polinomial}
Queremos predecir valores de la primera en función de valores observados
de la segunda.El análisis de regresión consiste en ajustar un modelo
a los datos, estimando coeficientes a partir de las observaciones, con 
el fin de predecir valores de la variable de respuesta a partir de una 
(regresión simple) o más variables (regresión multiple) predictivas o 
explicativas.\par
\indent El análisis de regresión se puede usar para:
\begin{itemize}
    \item identificar a las variables predictivas relacionadas con una variable de respuesta
    \item describir la forma de la relación entre estas variables y para derivar una función matemática óptima que modele esta relación
    \item predecir la variable de respuesta a partir de la(s) explicativas o predictoras
\end{itemize}
\subsection{Definiciones}
\begin{itemize}
    \item $M =$ grado del polinomio interpolante.
    \item $N =$ cantidad de datos.
    \item $X_{i} = (x_{i}^{0},...,x_{i}^{M}).$
    \item $ W = (w_{0},...,w_{M}).$
    \item $ Y(X,W) = XW^{T}.$
    \item $E_{D}(W) = (1/2)  \sum_{i=1}^{N} (Y(X_{i},W)-t_{i})^{2}$
    \item $E(W) = (1/2)  \sum_{i=1}^{N} (Y(X_{i},W)-t_{i})^{2} + (\lambda/2)||W||_{2}$
    \item $E_{RMS}(W) = \sqrt{2E(W)/N}$.
    \item $E_{DRMS}(W) = \sqrt{2E_{D}(W)/N}$.
\end{itemize}

Dado que la función $E(W)$ es convexa en w, está posee un
minimo y esté es único.\par
\subsection{Objetivo}
\indent Buscar cual es el W que minimice los errores de
$Y(X_{i},W)-t_{i}$.
\section{Boston Housing Dataset}
The dataset contains a total of 506 cases.\par
\indent Hay 14 atributos en cada caso del conjunto de datos.
\begin{itemize}
    \item CRIM - tasa de criminalidad per cápita por ciudad
    \item ZN - proporción de tierra residencial dividida en zonas para lotes de más de 25,000 pies cuadrados.
    \item INDUS - proporción de acres de negocios no comerciales por ciudad.
    \item CHAS - Variable ficticia del río Charles (1 si el tramo limita con el río; 0 si no)
    \item NOX - concentración de óxidos nítricos (partes por 10 millones)
    \item RM - número medio de habitaciones por vivienda
    \item AGE - proporción de unidades ocupadas por sus propietarios construidas antes de 1940
    \item DIS - distancias ponderadas a cinco centros de empleo de Boston
    \item RAD - índice de accesibilidad a carreteras radiales
    \item TAX - tasa del impuesto sobre el valor total de la propiedad por cada 10,000
    \item PTRATIO - número de alumnos por docente y por ciudad
    \item B - 1000(Bk - 0.63)$^{2}$ donde Bk es la proporción de negros por ciudad
    \item LSTAT - menor condición de la población
    \item MEDV - Valor medio de las viviendas ocupadas por sus propietarios en miles de dólares.
\end{itemize}

\chapter{Desarrollo}
El desarrollado de este trabajo pretende hacer un análisis de como afecta
la concideración de diferentes funciones de error al vector solución y la 
variación de la media y varianza muestrales de los errores.

\section{Regresión Polinomial}
Se implementan funciones de error.\par
\indent Se realiza un analisis de el valor medio y la desviación 
standard muestrales para cada grado de polinomio considerado.\par
\indent Graficamos la influencia del grado del polinomio
sobre el $E_{RMS}$
\section{Boston Housing Dataset}
Lo que se busca es mostrar las relaciones entre
las features de este set de datos.\par
\indent Para ellos graficaremos los scatter-plot de todos
los features y la matriz de correlación.\par
\indent Luego para aquellas features que presenten
un coeficiente de correlación con la feature MEDV mayor
a 0.5 en valor absoluto se ajustan con polinomios de grado
1, 2 y 3, se grafican los scatter-plot correspondientes
y se calculan los $E_{RMS}$.


\chapter{Experimentación}
\section{variación de los grados de Polinomios}


\begin{figure}[h]
    \centering
    \includegraphics[width=\textwidth]{1_6.png}
    \caption{}
    \label{}
\end{figure}
\begin{figure}[h]
    \centering
    \includegraphics[width=\textwidth]{2_4.png}
    \caption{}
    \label{}
\end{figure}
\begin{figure}[h]
    \centering
    \includegraphics[width=\textwidth]{2_9.png}
    \caption{}
    \label{}
\end{figure}

\begin{figure}[h]
    \centering
    \includegraphics[width=\textwidth]{error_INDUS_lambda.png}
    \caption{}
    \label{error_INDUS_lambda}
\end{figure}
\begin{figure}[h]
    \centering
    \includegraphics[width=\textwidth]{error_INDUSNOXRMAGETAX_lambda.png}
    \caption{}
    \label{error_INDUSNOXRMAGETAX_lambda}
\end{figure}
\begin{figure}[h]
    \centering
    \includegraphics[width=\textwidth]{error_NOX_lambda.png}
    \caption{}
    \label{error_NOX_lambda}
\end{figure}




\begin{figure}[h]
    \centering
    \includegraphics[width=\textwidth]{scatter.png}
    \caption{Scatter-plots de Boston Housing Dataset}
    \label{scatter}
\end{figure}
\begin{figure}[h]
    \centering
    \includegraphics[width=\textwidth]{matriz_correlacion.png}
    \caption{Matriz de correlación de Boston Housing Dataset}
    \label{matriz_correlacion}
\end{figure}
\begin{figure}[h]
    \centering
    \includegraphics[width=\textwidth]{rmXmedv.png}
    \caption{Estimación de la distribución de puntos entre MEDV y RM}
    \label{rmXmedv}
\end{figure}
\begin{figure}[h]
    \centering
    \includegraphics[width=\textwidth]{ptratioXmedv.png}
    \caption{Estimación de la distribución de puntos entre MEDV y PTRATIO}
    \label{ptratioXmedv}
\end{figure}
\begin{figure}[h]
    \centering
    \includegraphics[width=\textwidth]{lstatXmedv.png}
    \caption{Estimación de la distribución de puntos entre MEDV y LSTAT}
    \label{lstatXmedv}
\end{figure}
\begin{figure}[h]
    \centering
    \includegraphics[width=\textwidth]{errorRM.png}
    \caption{Media y Varianza al estimar la relación entre MEDV y RM}
    \label{errorRM}
\end{figure}
\begin{figure}[h]
    \centering
    \includegraphics[width=\textwidth]{errorPTRATIO.png}
    \caption{Media y Varianza al estimar la relación entre MEDV y PTRATIO}
    \label{errorPTRATIO}
\end{figure}
\begin{figure}[h]
    \centering
    \includegraphics[width=\textwidth]{errorLSTAT.png}
    \caption{Media y Varianza al estimar la relación entre MEDV y LSTAT}
    \label{errorLSTAT}
\end{figure}

\chapter{Conclusiones}
Luego de realizar todos los experimentos y haber entendido bien el funcionamiento, podemos realizar algunas conclusiones.

\section{Referencias}
1. Pattern Recognition and Machine Learning by C. Bishop, Springer 2006. 

\end{document} 