El desarrollado de este trabajo pretende hacer un análisis de como afecta
la concideración de diferentes funciones de error al vector solución y la 
variación de la media y varianza muestrales de los errores.

\section{Regresión Polinomial}
Se implementan funciones de error.\par
\indent Se realiza un analisis de el valor medio y la desviación 
standard muestrales para cada grado de polinomio considerado.\par
\indent Graficamos la influencia del grado del polinomio
sobre el $E_{RMS}$
\section{Boston Housing Dataset}
Lo que se busca es mostrar las relaciones entre
las features de este set de datos.\par
\indent Para ellos graficaremos los scatter-plot de todos
los features y la matriz de correlación.\par
\indent Luego para aquellas features que presenten
un coeficiente de correlación con la feature MEDV mayor
a 0.5 en valor absoluto se ajustan con polinomios de grado
1, 2 y 3, se grafican los scatter-plot correspondientes
y se calculan los $E_{RMS}$.
