

\section{Regresión polinomial}
Queremos predecir valores de la primera en función de valores observados
de la segunda.El análisis de regresión consiste en ajustar un modelo
a los datos, estimando coeficientes a partir de las observaciones, con 
el fin de predecir valores de la variable de respuesta a partir de una 
(regresión simple) o más variables (regresión multiple) predictivas o 
explicativas.\par
\indent El análisis de regresión se puede usar para:
\begin{itemize}
    \item identificar a las variables predictivas relacionadas con una variable de respuesta
    \item describir la forma de la relación entre estas variables y para derivar una función matemática óptima que modele esta relación
    \item predecir la variable de respuesta a partir de la(s) explicativas o predictoras
\end{itemize}
\subsection{Definiciones}
\begin{itemize}
    \item $M =$ grado del polinomio interpolante.
    \item $N =$ cantidad de datos.
    \item $X_{i} = (x_{i}^{0},...,x_{i}^{M}).$
    \item $ W = (w_{0},...,w_{M}).$
    \item $ Y(X,W) = XW^{T}.$
    \item $E_{D}(W) = (1/2)  \sum_{i=1}^{N} (Y(X_{i},W)-t_{i})^{2}$
    \item $E(W) = (1/2)  \sum_{i=1}^{N} (Y(X_{i},W)-t_{i})^{2} + (\lambda/2)||W||_{2}$
    \item $E_{RMS}(W) = \sqrt{2E(W)/N}$.
    \item $E_{DRMS}(W) = \sqrt{2E_{D}(W)/N}$.
\end{itemize}

Dado que la función $E(W)$ es convexa en w, está posee un
minimo y esté es único.\par
\subsection{Objetivo}
\indent Buscar cual es el W que minimice los errores de
$Y(X_{i},W)-t_{i}$.
\section{Boston Housing Dataset}
The dataset contains a total of 506 cases.\par
\indent Hay 14 atributos en cada caso del conjunto de datos.
\begin{itemize}
    \item CRIM - tasa de criminalidad per cápita por ciudad
    \item ZN - proporción de tierra residencial dividida en zonas para lotes de más de 25,000 pies cuadrados.
    \item INDUS - proporción de acres de negocios no comerciales por ciudad.
    \item CHAS - Variable ficticia del río Charles (1 si el tramo limita con el río; 0 si no)
    \item NOX - concentración de óxidos nítricos (partes por 10 millones)
    \item RM - número medio de habitaciones por vivienda
    \item AGE - proporción de unidades ocupadas por sus propietarios construidas antes de 1940
    \item DIS - distancias ponderadas a cinco centros de empleo de Boston
    \item RAD - índice de accesibilidad a carreteras radiales
    \item TAX - tasa del impuesto sobre el valor total de la propiedad por cada 10,000
    \item PTRATIO - número de alumnos por docente y por ciudad
    \item B - 1000(Bk - 0.63)$^{2}$ donde Bk es la proporción de negros por ciudad
    \item LSTAT - menor condición de la población
    \item MEDV - Valor medio de las viviendas ocupadas por sus propietarios en miles de dólares.
\end{itemize}